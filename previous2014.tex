\section{Results of Previous NSF-Funded Research}\label{SectPrevious}

Greg Fasshauer (GEF) and Fred Hickernell (FJH) are the PIs for \emph{NSF-DMS-1115392 Kernel Methods for Numerical Computation}, July 1, 2011 -- June 30, 2014, \$320,000.  Since Fall 2005, we have organized a weekly Meshfree Methods research seminar that draws regular participation from all of our local collaborators, including faculty members, visitors, and students.  All participants take turns posing interesting research problems, reporting work-in-progress, and presenting relevant work of others.  The atmosphere is informal; interruption and discussion is encouraged.  Occasionally we have speakers from outside applied mathematics and/or IIT. Regular participants for a month or more since July 1, 2011 include the following (``F'' = female,  ``A'' = African-American, ``L'' = Latina/Latino).

\begin{itemize}[leftmargin=2.5ex]
\item IIT Faculty: GEF, FJH, Igor Cialenco (IC), John Erickson (JE, A), and Lulu Kang (LK, F);
\item Long-Term Visitors: Roberto Cavoretto (RC), University of Torino; Sou-Cheng Choi (SCC, F), University of Chicago; Salvatore Ganci (SG), University of Palermo; YungWook Jung, Gyeonggi College of Science and Technology; Junbin Li, Dalian University of Technology; Yiuwei Liu (YiL), Lanzhou University; Michael McCourt (MM), Cornell University and Argonne National Laboratory; Jinming Wu, Zhejiang Gongshang University;
\item IIT Graduate Students: Aleks Borkovskiy (AB), Siyuan Deng (SD), Yuhan Ding (YD, F), Llu\'is Antoni Jim\'enez Rugama (LAJR), Lan Jiang (LJ, F), Yao Li (YaL), Yiou Li (YL, F), Jagadeeswaran Rathinavel (JR), Tiago Silva (TS), Xin Tong (XT, F), Qi Ye (QY), Xiaodong Zhang (XiZ), Yizhi Zhang (YZ), Xuan Zhou (XZ);
\item IIT Undergraduate Students: Haocheng Bian (HB), Nick Clancy (NC), Martin Dillon (MD), Caleb Hamilton (CH), Joseph Kupiec, Barrett Leslie (BL), Martha Razo (MR, L, F);
\item Other Undergraduate Students: Casey Bylund (CB, F), University of San Francisco; Matthew Gliebe (MG), Northwestern University; William Mayner (WM), Brown University;
\item High School Student: Sunny Yun (SY).
\end{itemize}

\subsection{Intellectual Merit from Current NSF Funding}

Together with a collaborator, we have investigated the convergence and tractability of meshfree methods using the isotropic Gaussian kernel and its extension with non-isotropic shape parameters, $\gamma_{\ell}$, \eqref{gausskernel}. Dimension \emph{independent} convergent rates for function approximation are possible and dependent on the rate of decay of the $\gamma_{\ell}$ \citep{FasHicWoz12b, FasHicWoz12a}. These results have been extended by XZ to a more general class of kernels with scale and shape parameters \citep{ZhoHic15a}.

Together with QY and other collaborators, we have shown how the optimal solution of problems posed on reproducing kernel Hilbert spaces can be extended to Banach spaces \citep{SonZhaHic12a, FasHicYe13a}.

Solution of more complex multivariate or even infinite-dimensional approximation and integration problems require algorithms with parameters that must be hand-tuned, e.g., based on the shape parameters of the kernel.  We believe that users need automatic algorithms requiring a minimum of tuning, but still justified by theory.  This has led us to backtrack to look at simpler problems and algorithms.  FJH, JL, YiL, and a collaborator have developed a guaranteed automatic algorithm for Monte Carlo calculation of means of random variables and of multidimensional integrals \citep{HicEtal14a}, as described in Sect.\ \ref{MC_g_sec}. This work first showed the importance of the cone idea on which Sect.\ \ref{SectCones} is based. Other scenarios under investigation include guaranteed quasi-Monte Carlo error estimation (FJH and LAJR), and guaranteed multi-level Monte Carlo (FJH, AB and YaL).

FJH, NC, YD, CH, and YZ applied the cone idea to construct guaranteed, adaptive, deterministic algorithms for univariate integration and function approximation \citep{HicEtal14b}, again using the idea of cones as described in Sect.\ \ref{integral_g_sec}.  This paper also sets out a general framework for constructing guaranteed automatic algorithms for any problem where the solution operator is homogeneous.  FJH, MR, and SY are preparing a paper to explain these ideas to a broader audience.

FJH, YD, LJ, and YZ have implemented the algorithms constructed by \cite{HicEtal14a} and \cite{HicEtal14b} in GAIL \citep{ChoiEtal13a} as mentioned in Sect.\ \ref{SectCones}.  This \Matlab library is publicly available and under ongoing development.

GEF and QY have formulated a theory of generalized Sobolev spaces for reproducing kernel Hilbert spaces that connects kernel-based approximation methods to fundamental solutions of differential operators \cite{FasshauerYe11, FasshauerYe13}. An important contribution is the fact that the definition of these generalized Sobolev spaces involves both a notion of smoothness and of scale, while traditional Sobolev spaces take into account only the smoothness of functions in the space.

As an application of the generalized Sobolev space theory, GEF, QY and IC have developed a framework for the numerical solution of stochastic partial differential equations with kernel-based methods \citep{CFY12,FasshauerYe13b,FasshauerYe14}.

GEF and MM have developed an RBF-QR algorithm for the stable computation with ``flat'' kernels as explained in Sect.~\ref{SectRBFQR}. This approach takes advantage of the Hilbert-Schmidt series of the kernels \eqref{HSseries}. The basic framework was reported in a paper on stable computations with Gaussian kernels \citep{FMcC12}, and further extensions to the solution of boundary value problems for PDEs \citep{McCourt13}, a fast recursive regression algorithm \citep{McCourt13b}, and coupled PDE problems \citep{McCF13} have been submitted for publication. A publicly available library of \Matlab code \citep{McCFBG13} is under ongoing development.

A new class of so-called \emph{compact Mat\'ern kernels} was investigated by RC, GEF and MM \citep{CFMcC13}. The paper discusses these kernels, their connection to piecewise polynomial splines and an implementation of the associated Hilbert-Schmidt SVD algorithm. Special cases of these kernels were studied by CB and WM during their 2012 summer REU under the guidance of GEF and MM. During summer 2013, GEF and MM worked with two other undergraduate REU students, HB and MG, on the numerical computation of kernel eigenvalues and eigenfunctions and on an application of the Hilbert-Schmidt SVD to the study of the Hilbert space norm of the kernel interpolant. The latter problem is important for the error estimation of kernel-based methods, and for the determination of optimal kernel parameters such as the shape parameter.

During SG's visit to IIT (08/2012-02/2013) he performed joint work with GEF and MM on the use of kernel-based methods for the solution of coupled boundary value problems for the application in EEG and MEG (see Sect.~\ref{SecMEEG}). This work made use of the coupling framework of \cite{McCF13}. A joint paper \citep{AFFGM13} has been submitted.

Understanding the covariance kernels, eigenfunctions and associated Hilbert spaces of certain stochastic processes, in particular fractional Brownian motion, was the focus of a research project during summer 2013 involving MD under the guidance of JE and GEF. Some outcomes of this research will be presented at the 2014 Joint Mathematics Meeting.

\subsection{Broader Impacts from Current NSF Funding}

We have trained a number of PhD, MS, BS, and high school students in research, as evidenced by the list above.  QY defended his PhD thesis in April 2012 and became a Philip T. Church postdoc in the Department of Mathematics at Syracuse University starting August 2012. He has four published papers, with an additional two accepted, one submitted, and two more in preparation.  Of the other ten IIT PhD students listed, YD, LJ, YL, and YZ have completed their comprehensive exams and are expected to graduate by the end of 2014.  AB and XZ have passed their qualifying exams and will take their comprehensive exams within 2014.  LAJR, JR, TS, and XiZ are in the earlier stages of their PhD studies. SD, YaL, TS, and XiZ completed their MS theses. The latter two are now PhD students.  QY, YD, LJ, LAJR, YL, XZ, HB, MR, SY, NC, MD, CH, BL, CB, WM, and SY have all presented their work at academic meetings in talks or posters or will do so within the next four months.

We have been pleased to have two underrepresented minority scientists and several female scientists (undergraduate through to faculty members) in our research group.  To their credit, both LJ and YL have successfully pursued their PhDs while at the same time carrying the responsibility of marriage and motherhood.

GEF presented a week-long tutorial to students from many different engineering disciplines at the University of Palermo. As a result, SG, a PhD student from Palermo supported by Italian funds, visited IIT from 08/2012-02/2013 to receive guidance on his research from GEF and MM.

RC, a postdoctoral researcher from the University of Torino supported by Italian funds, visited IIT from 03/2012-05/2012 and also in April 2013. He received advanced training in the area of meshfree approximation and has since been able to secure a position as a research fellow at the University of Torino.

Every fall semester in even years GEF teaches the graduate course MATH 590: Meshfree Approximation Methods, which focuses on kernel based methods, and uses his monograph \citep{Fas07a} as a text. The new developments using eigenfunction expansions of positive definite kernels were integrated for the first time in fall 2012. GEF and MM have used the updated lecture notes of this course as a basis for a new book manuscript entitled ``Kernel-based Approximation Methods using \Matlab'', currently under contract with World Scientific Publishers.

Every fall semester FJH teaches the graduate course MATH 565: Monte Carlo Methods in Finance.  The new results on automatic Monte Carlo algorithms have been taught there since fall 2012, and preliminary results on multi-level Monte Carlo have been taught since fall 2013.

SCC and FJH offered an experimental graduate seminar course for seven students, MATH 573: Reliable Mathematical Software, during fall 2013. Our goal was to help computational mathematicians understand how to develop software that can be used by others.  Topics included guarantees for automatic algorithms, reproducible computational science research, efficient coding, thorough documentation, careful testing, convenient user interfaces with parameter parsing and validation, and software publication.  We intend to offer this course again to an expanded audience because the topics are not normally covered in standard mathematics or computer science courses.

Members of the research team have organized conferences. Together with Larry L.~Schumaker, GEF organized the NSF-supported 14th International Conference on Approximation Theory in San Antonio, TX, in April 2013. A special mentoring session for PhD students and postdocs was included in the program of this conference. QY was the main organizer for the SIAM Student Chapter Conference on Recent Advances in Computational Science and Statistics held at IIT Oct. 29--30, 2011 ({\tt http://www.math.iit.edu/$\sim$siam/workshop/}). BL was a co-organizer. This conference brought together graduate students and postdocs from various Chicago area universities and Argonne National Lab and featured three plenary lectures (by Jerry L. Bona, Charles K. Chui, and Wei B. Wu) along with 25 contributed talks and a poster session. Since then other Chicago-Area SIAM chapters have joined the IIT chapter and organized additional joint conferences.

The research team has attended conferences, presented talks at departmental colloquia, and given talks to general audiences, both inside the mathematical sciences and beyond. Here are the highlights of the PIs.

\begin{itemize}[leftmargin=2.5ex]
\item GEF gave minisymposium talks at the 2012 and 2013 SIAM Annual Meetings.  He also gave invited presentations at the workshop on Multivariate Approximation and Interpolation with Applications in Erice, Italy in Sept.~2013, at the Freeform Optics Incubator Meeting of the Optical Society of America, Washington, D.C., in Oct.~2011, and at the NSF-CBMS Conference on Radial Basis Functions: Mathematical Developments and Applications, at UMass-Dartmouth, in June 2011. He gave a talk to the Level Set Collective at the NSF-sponsored Institute of Pure and Applied Mathematics (IPAM) and an invited talk at the 2013 Midwest Numerical Analysis Day. GEF is also a steering committee member of the Midwest Numerical Analysis Conference series. In 2011, he gave a week-long workshop at the U of Palermo (Italy, engineering). GEF gave departmental colloquium/seminar talks at Chapman U (math \& computer science), U Chicago (statistics), Middle Tennessee State U (math), Northwestern U (industrial engineering \& management sciences), and U of Padua (Italy, math).  GEF attended the 2010-2012 Intel International Science and Engineering Fairs (2011-2012 as chair of the AMS Menger Prize committee), where he judged projects in the mathematical sciences and interviewed high school students about their research. These interviews can be viewed on the YouTube channel of the AMS. He has also given presentations to students at Neuqua Valley and Carl Sandburg high schools.

\item FJH gave a talk at a special session at the 2012 Joint Mathematics meetings, a plenary lecture at the Tenth International Conference on Monte Carlo and Quasi-Monte Carlo Methods in Scientific Computing in 2012, and an invited talk at the 2013 Midwest Numerical Analysis Day.  He will speak at the 2014 Joint Mathematics meetings and give an invited session talk at the Eleventh International Conference on Monte Carlo and Quasi-Monte Carlo Methods in Scientific Computing in April, 2014. FJH gave departmental colloquium/seminar talks at Argonne National Laboratory (mathematics and computer science), U Chicago (statistics), DePaul U (mathematics), Georgia Tech (industrial and systems engineering), and Illinois Institute of Technology (computer science, mathematics).  FJH was invited to speak to STEM students at Oakton Community College and to students at Hinsdale Central High School.

\end{itemize}
