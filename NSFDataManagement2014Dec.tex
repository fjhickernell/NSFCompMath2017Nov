% !TEX TS-program = PDFLatexBibtex
%&LaTeX
%Greg and Fred's NSF Grant Dec 2014
\documentclass[11pt]{NSFamsart}
\usepackage{latexsym,amsfonts,amsmath,epsfig,multirow,stackrel,natbib,tabularx,enumitem,xspace}

% This package prints the labels in the margin
%\usepackage[notref,notcite]{showkeys}


%\pagestyle{empty}
\thispagestyle{plain}
\pagestyle{plain}

\headsep-0.6in
%\headsep-0.45in

\setlength{\textwidth}{6.5in}
\setlength{\oddsidemargin}{0in}
\setlength{\evensidemargin}{0in}
\textheight9in
%\textheight9.1in



%\setcounter{page}{1}


\begin{document}
%\setlength{\leftmargini}{2.5ex}

\centerline{\textbf{\Large Data Management Plan}}

\bigskip



This plan will make certain that the data produced during the period of this project is appropriately managed to ensure its usability, access and preservation.  The data produced by the proposed project will consist of theory, numerical software, and potential coursework. 

The PIs will publish the results of their theoretical, and computational investigations as early as appropriate in the form of peer-reviewed journal articles, conference abstracts and talks at various conferences and institutions. Authorship will accurately reflect the contributions of those involved.  Students will be particularly encouraged to publish their work. Software resulting from this project of a general nature will be published on a publicly available website and be publicized through talks and e-newsletters like the NA-Digest.

We will fully comply with all applicable guidelines and policies on model and data sharing as mandated or recommended by NSF.

This Data Management Plan addresses NSF’s policy on the dissemination and sharing of research results within a reasonable time.  In accordance with this policy, this plan does not include preliminary analyses (including raw data), drafts of scientific papers, plans for future research, peer reviews, or communications with colleagues. 



\end{document}
