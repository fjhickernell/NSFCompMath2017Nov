% !TEX TS-program = PDFLatexBibtex
%&LaTeX
%Greg and Fred's NSF Grant Dec 2014
\documentclass[11pt]{NSFamsart}
\usepackage{latexsym,amsfonts,amsmath,epsfig,multirow,stackrel,natbib,tabularx,enumitem,xspace}

% This package prints the labels in the margin
%\usepackage[notref,notcite]{showkeys}


%\pagestyle{empty}
\thispagestyle{plain}
\pagestyle{plain}

\headsep-0.6in
%\headsep-0.45in

\setlength{\textwidth}{6.5in}
\setlength{\oddsidemargin}{0in}
\setlength{\evensidemargin}{0in}
\textheight9in
%\textheight9.1in



%\setcounter{page}{1}


\begin{document}
%\setlength{\leftmargini}{2.5ex}

\centerline{\textbf{\Large Data Management Plan}}

\bigskip



This plan will make certain that the data produced during the period of this project is appropriately managed to ensure its usability, access and preservation.  The data produced by the proposed project will consist of theory, numerical simulations and experimental measurements of the Industry Prompted Projects, as well as material for the related coursework.  If needed, computer software will be produced as well. 

Investigators will jointly develop a project website at IIT to further collaboration between the MATHub and our industry partners.  All simulation and experimental data will be retained indefinitely (for a period of no less than 5 years) on IIT servers.  Publications of the results of the theoretical, numerical and experimental investigations will occur as early as appropriate in the form of peer-reviewed journal articles, conference abstracts and talks at various institutions. Authorship will accurately reflect the contributions of those involved.  Trainees will be particularly encouraged by the investigators to publish their work.  Data will be made accessible immediately after publication.  Should patent applications arise from the results of the investigations, publication of data might require longer times as appropriate and determined by Intellectual Property and Legal Offices at IIT . 

The project website will be designed to have a public section, open to anyone, and a private section.  The public section will include selected data sets, publications from the project, and movies of the simulations.  A private website will be password protected, and be available for distance collaboration between the researchers. This website will include links to the previously mentioned repositories of the software. 

We will fully comply with all applicable guidelines and policies on model and data sharing as mandated or recommended by NSF.  In general, data will be shared with collaborators as soon as possible, with local colleagues at seminars and talks, and with the scientific community at large by posters and presentations at meetings, by peer-reviewed publication to the widest audience possible, and by the project website.  In some cases, requestors will be asked to sign a formal data sharing agreement, which will provide for a commitment to use data only for research purposes. 

This Data Management Plan addresses NSF’s policy on the dissemination and sharing of research results within a reasonable time.  In accordance with this policy, this plan does not include preliminary analyses (including raw data), drafts of scientific papers, plans for future research, peer reviews, or communications with colleagues.  Furthermore, data to enable peer review and publication/dissemination and/or to protect intellectual property may be temporarily withheld from distribution and other proposed data management. 



\end{document}
