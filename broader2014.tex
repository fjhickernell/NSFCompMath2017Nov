\section{Broader Impact of Proposed Research}\label{SectBroad}


\subsection{Contributions to Training, Mentoring and Other Human Resource Developments}

The PIs, GEF and FJH, are dedicated to mentoring young researchers from high school through postdoctoral levels.  We are excited to see our mentees transition from curiosity about what scholarship entails to choosing and solving their own problems.  We are committed to our regular Meshfree Methods research seminar because we believe that we learn good research practices by seeing them modeled and we that find good research ideas from a variety of sources. Outside of the Meshfree Methods seminar we meet with members individually or in smaller groups to discuss details of their research projects.

\begin{description}[leftmargin=2.5ex]
\item[Providing Research Experiences for Undergraduates and High School Students]\ We \linebreak[4] strongly believe that students should be introduced to research before graduate school so that they can learn how to discover the unknown, something that is not taught well in a classroom. We request funds to support two summer REU students per year. Having advertised our REU opportunities for several years now has given us some visibility within the community and is prompting inquiries from prospective participants well before we even announce our latest program offerings. As in the past two years, we expect the NSF funds will serve as a catalyst for funds to support additional summer students. In choosing REU students we make a deliberate effort to build a diverse research environment by targeting female and underrepresented minority students as well as students from less research-focused institutions (see Sect.~\ref{SectPrevious}). We will also continue to receive well-prepared high school students to join our research group as we did last summer.

\item[Preparing Students for Academic Careers] We consider mentoring to be a multi-faceted and potentially long-term process continuing even after the mentee has moved on from IIT.  For example, MM was an undergraduate student at IIT who collaborated with GEF during his PhD studies at Cornell University.  We will continue to mentor him as senior personnel for this proposal. Similarly, although QY has left IIT, GEF serves as his designated mentor on his application for travel support to the 2014 ICM in Korea. We have provided and will continue to provide SCC, a post-doctoral scientist and another senior personnel for this project, teaching and mentoring experience by including her in our present NSF project.  We will continue to find opportunities for special mentoring activities for our students, like those GEF organized and that QY and YD were involved in during the recent NSF-supported 14th International Conference on Approximation Theory in San Antonio, TX.  We  will continue our collaboration with Argonne National Laboratory (ANL), which has led to short-term and long-term opportunities for our PhD students and graduates.

\item[Preparing Students for Industry Careers]
In addition to preparing students for the academic landscape, we also help current students land competitive jobs in the business world. The training we provide in the areas of algorithm development and coding tends to give our students the needed edge. For example, WM, who graduated from Brown University in May 2013 is currently working for an internet startup company.

\item[Supervising Visitors]
GEF has established contacts with several Italian universities attracting students and postdoctoral visitors to IIT for extended visits (see Sect.~\ref{SectPrevious}). A visit by a PhD student from the University of Padua is currently in the planning stages for the fall of 2014.  Having lived in Hong Kong for 19 years, FJH has contacts with Chinese scientists that have prompted several long-term visiting Chinese scholars and students to join our research.  These activities will continue.

\item[Giving Short Courses and Invited Lectures]
We will continue our active track record (see Sect.~\ref{SectPrevious}) of providing lectures to students at various stages in their careers, ranging from high school to graduate school. These encourage students to enter STEM and encourage STEM students to engage in research.

\end{description}

\subsection{Contributions to Resources in Research, Education and the Broader Society}

The research we propose straddles mathematics, statistics, computer science, and applications in engineering and related fields.  The two PIs have complementary strengths that facilitate this interdisciplinary research.  GEF has expertise in approximation theory, meshfree methods, and numerical partial differential equations, while FJH has expertise in (quasi-)Monte Carlo methods, kernel-based methods, information-based complexity theory, and experimental design. Our expertise provides both an obligation and an opportunity to interact with a number of diverse communities. We envision the following contributions:

\begin{description}[leftmargin=2.5ex]
\item[Disseminating Research]
The research supported by this grant will result in publications in peer-reviewed journals in a broad spectrum of applied mathematics, computer science, statistics and engineering. These journals will include both those that emphasize theory and those that emphasize applications.

\item[Promoting Cones] The idea of guaranteed, adaptive algorithms via cones of input functions has broad potential application.  We will be promoting this idea among numerical analysts who develop new algorithms and analyze their computational costs, as well as among information-based complexity theorists who analyze the lower bounds on the complexity of numerical problems.

\item[Promoting the Hilbert-Schmidt SVD] Similarly, we will encourage other researchers to take advantage of the stability given to kernel-based methods by the Hilbert-Schmidt SVD, to use our code, and to join our research efforts.

\item[Bridging Mathematics and Statistics]
This project touches on topics that are of interest to the statistics community: kriging, Monte Carlo methods, and design of experiments.  Historically, there has been relatively little interaction between numerical analysts and statisticians.  We have and will continue to engage the statistics community by speaking a their conferences and departmental seminars.  MM is currently delivering a series of lectures to the statistics group at CU-Denver, and GEF will join this effort during several visits during the spring 2014 semester.

\item[Collaborating with Engineers]
As a result of the visit by SG (see Sect.~\ref{SectPrevious}), GEF and MM have recently submitted a bilateral research proposal to the Italian government together with Guido Ala (Dept. Electrical Engineering, U Palermo) and Elisa Francomano (Dept. Chemical Engineering, Management, Computer Science and Mechanical Engineering, U Palermo) entitled ``Novel Numerical Methods and Embedded Systems for the Integration of Neuroimaging Techniques''. In that project kernel methods are being used for the solution of inverse problems arising in the detection of brain activity from MEG or EEG data (see Sect.~\ref{SecMEEG}). This fully non-invasive diagnostic procedure will enable doctors to detect early functional and neurophysiological markers of diseases (e.g., of Alzheimer's disease). It will also result in a potential reduction in doctors' visits, shorter hospitalization periods and a greater longevity with overall improved quality of life. The bilateral project proposal does not include any direct support for GEF and MM. In fact, it depends on travel funds obtained via the current proposal. Since kernel methods are becoming a rather popular numerical tool in science in engineering, other opportunities for collaborations outside mathematics frequently arise. For example, GEF participated in an incubator meeting on freeform surfaces organized by the Optical Society of America.

\item[Organizing and Presenting at Conferences]
We and our students involved in this project will present our results at a variety of conferences and workshops.  These include: (i) specialized meetings focusing on approximation theory, complexity, experimental design, meshfree methods, and Monte Carlo methods; (ii) the national meetings of AMS, SIAM, and the statistical societies; and (iii) conferences devoted to application areas.  We are frequently invited to speak at such conferences, which will give our results a prominent hearing. We will also continue to organize specialized conferences or minisymposia within larger conferences. For example, GEF and MM are planning to organize a minisymposium on \emph{Advances in Kernel Methods for Analysis and Statistics} at the 2014 SIAM Annual Meeting, and FJH and SCC are organizing a minisymposium on \emph{Reliable Computational Science} at the same meeting.

\item[Writing Textbooks and Survey Papers]
GEF and MM are currently under contract with \linebreak[4] World Scientific Publishers to prepare a monograph entitled \emph{Kernel-based Approximation Methods using \Matlab}. This book will provide an exposition of the theory and implementation of the Hilbert-Schmidt SVD along with numerous applications. The book will form a bridge to the GaussQR library currently under development and may serve as a textbook for a graduate class on meshfree methods, such as MATH 590 at IIT. GEF and FJH occasionally publish survey articles (e.g., a 42 page paper on kernel-based methods \citep{Fasshauer11}).

\item[Refreshing Course Syllabi]
MATH 590 (Meshfree Methods), taught by GEF in the fall semester of every even-numbered year, is likely to undergo a major change in the fall of 2014 as the preparation of the monograph mentioned above progresses. MATH 565 (Monte Carlo Methods in Finance), taught every fall by FJH, already includes our new results on guaranteed confidence intervals using IID sampling, and in the future will include our work in progress on guaranteed multi-level and quasi-Monte Carlo sampling (Sect.\ \ref{MC_g_sec}).

\item[Changing How Numerical Analysis Is Taught] Current texts teach students to estimate the error of the trapezoidal rule, $T_n(f)$, by $[T_n(f)-T_{n/2}(f)]/3$ (see, e.g., \cite[p.\ 223--224]{BurFai10}).  The arguments leading to this estimate introduce the valuable concept of extrapolation, however, this is a flawed error estimate for the same reasons that the error estimate in \Matlab's {\tt integral.m} is flawed (see Sect.\ \ref{drugssubsect}).  We will urge numerical analysis textbook authors and educators to change the way that error estimation is taught based on our recent and proposed work.  These ideas will also enter our more traditional numerical analysis courses such as MATH 350 (Introduction to Computational Mathematics) or MATH 577 and 578 (Computational Mathematics I \& II).

\item[Creating Software and Collaborating with Software Developers]
GEF and MM have created the website \citep{McCFBG13} which serves as the home for the software on our stable algorithms built upon the Hilbert-Schmidt SVD. This \Matlab library is freely available and allows others to experiment with our code. Thus far, the library contains routines for Gaussian kernels and for compact Mat\'ern kernels. As our research expands to other kernels and their eigenexpansions the resulting code will be added to the library. The GaussQR library also serves as a sandbox for students---especially REU students---to learn about our research and allows them to contribute pieces of their own work.

We will continue to develop GAIL \citep{ChoiEtal13a} as part of our proposed research.  The GAIL software will serve the wider community that relies on numerical approximation and integration algorithms.  It will also demonstrate how automatic algorithms ought to be implemented, which we hope will inspire and inform those working on automatic, adaptive algorithms for other mathematical problems.  We also expect our new algorithms to be incorporated into widely used numerical packages, as was done for the algorithm in \cite{HonHic00a} by \Matlab \citep{MAT8.2} and NAG \citep{NAG23}.  We will continue to discuss with software developers about these issues.

\item[Reaching Out]
GEF and FJH both have a record of reaching out to high school students and we plan to continue. The website {\tt http://math.iit.edu/$\sim$openscholar/meshfree/} helps manage our internal and external communication and dissemination of research findings. Advertising for the summer REU experiences is also facilitated via this website.


\end{description}
